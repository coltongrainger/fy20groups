\documentclass[onesided]{ccg-pset}

\course{MATH 6270}
\psnum{3}
\author{Colton Grainger}
\date{\today}
\begin{document}
\maketitle

\begin{enumerate}
\bibliography{/home/colton/coltongrainger.bib}
\bibliographystyle{alpha}

\item 
For $p$ a prime, the \term{Prüfer $p$-group} $\Z(p^\infty) = \Z\paren{p^{\infty}}$
is the direct limit (or \term{colimit}) of the sequence of abelian groups
\begin{equation}
\label{colimseq}
0 \inj \Z/p \inj \Z/p^2  \inj \Z/p^3 \inj \cdots \inj \Z/p^n \inj \cdots
\end{equation}
where for each $n \in \N$ the quotient group $\Z/p^n$ is ``glued into'' $\Z/p^{n+1}$ by the $p$th power map.

Considering generators and relations (as in \cite{Rie17}), we can also describe the Prüfer $p$-group as
\begin{equation*}
    \Z/p^\infty := \ang{\set{g_1, g_2, g_3, \ldots} \mid \set{0, g_1p, g_2p^2, \ldots} }
\end{equation*}

If $H$ is a subgroup of $\Z(p^{\infty})$, then there either exists some $g_M \in H \cap \set{g_1, g_2, g_3, \ldots}$ of maximal exponent or not. 

Say $g_M$ of maximal exponent in $H$ does exist. I claim the inclusion of $H$ into $\Z(p^\infty)$ induces
\begin{equation*}
    \shortexact{H}{\Z(p^\infty)}{\Z(p^\infty)}{}
\end{equation*}
because $\Z(p^\infty)$ is isomorphic to the colimit of \eqref{colimseq} with $M$ additional trivial groups tacked onto the front of the sequence, i.e., the colimit of the diagram
\begin{equation}
\label{quotientedcolim}
    \underbrace{0 \inj \ldots \inj 0}_\text{$M+1$ trivial groups} \inj \Z/p \inj \Z/p^2 \inj \cdots
\end{equation}
This demonstrates that every finitely generated subgroup of $\Z(p^\infty)$ is a proper subgroup of $\Z(p^\infty)$. Hence $\Z(p^\infty)$ is not finitely generated.

Moreover, say $g_M$ of maximal exponent in $H$ does not exist. This with Neumann's theorem implies $H$ has a countably infinite generating set. Then for each $n \in \N$, the subgroup $H$ contains $\Z/p^n$. The universal property of the colimit pushes out a map that $\Z(p^\infty) \inj H$, hence knowing also $H \inj Z(p^\infty)$, we have an isomorphism of abelian groups.

So also we now have enough information to exhaustively list the subgroups of $\Z(p^\infty)$:
\begin{itemize}
    \item each cyclic group of prime power order $p^n$
    \item the trivial subgroup $0$
    \item the entire group $\Z(p^\infty)$ itself.
\end{itemize}

\end{enumerate}

\end{document}
