\documentclass[onesided]{ccg-pset}

\course{MATH 6270}
\psnum{1}
\author{Colton Grainger}
\date{\today}


\begin{document}
\maketitle

\begin{note}[Conventions and Symbols]
    \label{conventions}
    \hfill
    \begin{itemize}
        % \item Let functions apply from the right, just like TODO lists, or JavaScript's \texttt{pipe}%
        %     \footnote{%
        %         Example from \url{https://www.freecodecamp.org/news/pipe-and-compose-in-javascript-5b04004ac937/}.
        %     }
        %     operation:
% \begin{verbatim}
% pipe(
    % getName,
    % uppercase,
    % get6Characters,
    % reverse
% )({ name: 'Buckethead' });
% // 'TEKCUB'
% \end{verbatim}

        \item Let $\N$ denote the inductive set of \term{natural numbers}, i.e., the positive integers $\{1,2,3, \ldots\}$.
        \item Let $A\subset B$ and $H<G$ denote inclusions (not necessarily proper) in the categories $\Set$ (sets and functions) and $\Grp$ (group and homomorphisms) respectively.
        \item Let us work in the full subcategory $\mathsf{FinGrp}$ of finite groups, i.e., let us assume for the exercises below
            \begin{equation*}
                \text{if $G$ is a group, then $G$ is \term{finite}.}
            \end{equation*}
            \begin{itemize}
                \item Observe $\cat{FinGrp}$ injects faithfully into $\Grp$.
            \end{itemize}
        \item When a (finite) group $G$ has a \term{group action} on a set $S$, we may construct the following groups and sets.%
            \footnote{%
                Notation from \url{https://groupprops.subwiki.org/wiki/Burnside\%27s_lemma}.
            }
        
            \begin{itemize}
                \item For each $g \in G$, let 
                      \begin{equation}
                          \label{fix}
                          \Fix g := \Fix_S g := \set{x \in S : x \cdot g = x}
                      \end{equation}
                      be the set of points of $S$ \term{fixed} by $g$.
                \item For each $x \in S$, let
                    \begin{equation}
                        \label{orb}
                       \Orb x := \Orb_G x := \set{x \cdot g: g \in G} 
                    \end{equation}
                    be the set of points of $S$ in the \term{orbit} of $x$ under the action of $G$.
                \item For each $x \in S$, let 
                      \begin{equation}
                          \label{stab}
                          \Stab x := \Stab_G x := \set{g \in G : x \cdot g = x} 
                      \end{equation}
                      be the subgroup of $G$ that \term{stabilizes} $x$.
            \end{itemize}
        \item For a subgroup $H < G$ of the (finite) group $G$, let
            \begin{equation*}
                \core H := \core_G H := \bigcap_{g \in G} H^g
            \end{equation*}
            be the \term{normal core} of $H$ in $G$. 
            \begin{itemize}
                \item Observe $\core_G H$ is the kernel of the induced map $G \inj \Sym \set{Hg : g \in G}$ afforded by the regular action of $G$ on the cosets of $H$ in $G$.
            \end{itemize}
    \end{itemize}
\end{note}

\begin{enumerate}
    \item Given. Let $G$ be a group of order $2m$ for odd $m \in \N$. 

        To prove. There is a normal subgroup $H \nrel G$ of index $2$ and size $\abs{H} = m$.

\begin{proof}
    Since $2$ is a prime dividing the order of the group, $2 \mid \abs{G}$, Cauchy's theorem implies there exists an $a \in G$ such that $\ang{a}_G$ is a cyclic subgroup of $G$ of order $2$ and index $m$. Let $$\rho \colon G \inj \Sym \abs{G}$$ be the embedding afforded by the right regular action $G$ on itself (where $\rho_g \colon G \to G$ is an automorphism of sets for each $g \in G$). The image of $\ang{a}_G$ in $\Sym \abs{G}$ is $\set{\id, \rho_a}$, where $\rho_a^2 = \rho_{a^2} = \id$. Therefore the least common multiple%
        \footnote{%
            See Rotman, 1995, exercises 1.1.10 through 1.1.12.
        }
    of the cycle decomposition of $\rho_a$ is $2$, and hence $\rho_a$ is the product of disjoint simple transpositions---but how many? To see $\rho_a$ is exactly the product of $m$ disjoint simple transpositions, consider the fundamental counting principle for the restriction of the regular action to the subgroup $\ang{a}$ of $G$:
    \begin{equation*}
        \abs{\Stab_{\ang{a}} g} \cdot \abs{\Orb_{\ang{a}} g} = 2
    \end{equation*}
    Because $$\Stab_{\ang{a}} g = \set{\alpha \in \ang{a} : g \cdot \alpha = g} = \set{\alpha \in \ang{a} : \alpha = 1}$$ is trivial, we deduce each $g \in G$ belongs to an orbit of size $2$. Because $G$ is partitioned by $m$ of these $2$-orbits, $\rho_a$ is seen to be the disjoint union of $m$ simple transpositions. As $m$ is odd, $\rho_a$ has odd sign, which is enough to conclude that the composed group homomorphism
    \begin{equation*}
        G \xrightarrow{\rho} \Sym \abs{G} \xrightarrow{\text{sgn}} C_2
    \end{equation*}
    is surjective from $G$ onto the cyclic group of order $2$. The kernel of this homomorphism is a normal subgroup of $G$ of index $2$ and order $m$ (by the universal property of quotient groups).
\end{proof}

\item (Cauchy-Frobenius lemma) Given. Let $G$ be a finite group acting on a finite set $S$ with $n \in \N$ orbits. 

    To prove. 
    \begin{equation}
        \label{sum-of-fix}
        \sum_{g \in G} \abs{\Fix g} = \sum_{x \in S} \abs{\Stab x} = n \abs{G}.
    \end{equation}
        Therefore the number of orbits, i.e., the number of blocks in the partition of $S$ induced by the group action, can be computed by
    \begin{equation}
        \label{orbit-count}
        n = \frac{1}{\abs{G}} \sum_{g \in G} \abs{\Fix g}.
    \end{equation}
 
    \begin{proof}
            \footnote{%
            This proof is more or less from Week 6, Fall 2019, Algebra I, as taught by Prof. Nathaniel Thiem.
            } Consider the set \begin{equation}\sF = \set{(g,x) \in G \times S : x \cdot g = x},\end{equation} a subset of the finite Cartesian product $G \times S$. 
            Being finite, $\abs{\sF}$ may be computed by counting singletons, or equivalently by summing the cardinalities of slices while indexing over either $G$ or $S$. 
        Therefore
        \begin{equation}
            \sum_{g \in G} \abs{\set{x \in S: x \cdot g = x}} = \abs{F} = \sum_{x \in S} \abs{\set{g \in G: x \cdot g = x}}.
        \end{equation}
        By definition \eqref{fix}, $\Fix_S g = \set{x \in S: x \cdot g = x}$. 
        By definition \eqref{stab}, $\Stab_G x = \set{g \in G: x \cdot g = x}$. 
        Taking the previous three observations in stride,
        \begin{equation*}
            \sum_{g \in G} \abs{\Fix_S g} = \sum_{x \in S}\abs{\Stab_G x}.
        \end{equation*}
        By the fundamental counting principle, for each $x \in S$, $\abs{G}/\abs{\Orb_G x} = \abs{\Stab_G x}$. 
        Therefore
        \begin{equation*}
            \sum_{g \in G} \abs{\Fix_S g} = \abs{G}\sum_{x \in S} \frac{1}{\abs{\Orb_G x}}.
        \end{equation*}
        Observing that term $\sum_{x \in S} \frac{1}{\abs{\Orb_G x}}$ in the right hand side of the above equality is the number $n$ of disjoint orbits in $S$ induced by the action of $G$, the first proposition \eqref{sum-of-fix} follows. Dividing by $\abs{G}$ yields \eqref{orbit-count}.
    \end{proof}

\item Given. Let $G$ be a finite group, $p$ be a prime, and $\Syl_p G$ the set of all Sylow $p$-subgroups of $G$.

    To prove. 
    \begin{enumerate}
        \item The subgroup $O_p(G)$ defined by 
          \begin{equation}
              \label{def-o-p}
              O_p(G) := \bigcap \Syl_p G
          \end{equation}
              is \term{characteristic} in $G$.
          \item Moreover, $O_p(G)$ is the \term{unique largest normal $p$-subgroup} of $G$. That is, 
              \begin{equation}
                  \text{if $P \nrel G$ is a normal $p$-subgroup, then $P \nrel O_p(G)$.}
              \end{equation}
    \end{enumerate}

    \begin{proof}
        Sylow's theorem implies $G$ acts transitively (by conjugation) on $\Syl_p G$. Hence if $P_\text{syl} \in \Syl_p G$, then 
        \begin{equation*}
            O_p(G) = \bigcap_{g \in G}g^{-1} P_\text{syl} g = \core_G P_\text{syl}.
        \end{equation*}
        \begin{enumerate}
            \item  Because automorphisms of $G$ preserve the cardinalities of subgroups of $G$, the collection of Sylow $p$-subgroups (defined by their cardinality) are automorphism-conjugate. Because the $p$-core $O_p(G)$ is contained in each of the automorphism conjugate Sylow $p$-subgroups, $O_p(G)$ is invariant under automorphisms of $G$, hence characteristic. 
                  
            \item Let $P \nrel G$ be a normal $p$-subgroup. Sylow's theorem implies $P$ is a subgroup of some conjugate of $P_\text{syl}$. Because $P$ is invariant under inner automorphisms of $G$, we see that $P$ is contained in each conjugate of $P_\text{syl}$. Thence $P < O_p(G)$. Moreover, because normal subgroups satisfy a \term{transfer condition},%
                        \footnote{%
                            Defined here: \url{https://groupprops.subwiki.org/wiki/Transfer_condition}.
                        }
                    that $P \nrel G$ and $P < O_p(G)$ implies $P = P \cap O_p(G) \nrel O_p(G)$.
        \end{enumerate}
    \end{proof}

\item Given. Let $G$ be a finite group with $k$ conjugacy classes. Let $a_1, \ldots, a_k$ denote the orders of the centralizers of elements from the distinct classes.

    To prove.
    \begin{enumerate}
        \item The $a_j$ produce a ``finite sum of distinct unit fractions''%
            \footnote{%
                Quoted from \url{https://en.wikipedia.org/wiki/Egyptian_fraction}.
            }
        equal to unity:
          \begin{equation}
              \label{unity-rep}
              \frac{1}{a_1} + \frac{1}{a_2} + \ldots + \frac{1}{a_k} = 1.
          \end{equation}
      \item The finite groups with $3$ conjugacy classes or less are exactly those listed:
          \begin{itemize}
              \item Only the trivial group has exactly $1$ conjugacy class.
              \item Only the group $\Z/2\Z$ has exactly $2$ conjugacy classes. 
              \item Only the group $S_3$ has exactly $3$ conjugacy classes.
          \end{itemize}
    \end{enumerate}

    \begin{proof}
        Let $G$ act on itself by conjugation. The orbits in $G$ under the conjugation action partition $G$ into a disjoint union of conjugacy classes $G(x_1), \ldots, G(x_n)$ for some representative points $x_i \in G$. Since $G$ is finite, its order is computed by applying the fundamental counting principle and summing over the disjoint union:
        \begin{equation*}
            \abs{G} = \sum_{i=1}^n \abs{G: C_G(x_i)}.
        \end{equation*}
        
    \begin{enumerate}
        \item Dividing the above equation by $\abs{G}$ yields \eqref{unity-rep}. Suppose the $a_i$ have been arranged into an integer partition of $\abs{G}$,
            \begin{equation*}
                a_k \ge \cdots \ge a_2 \ge a_1 \qq{ with } a_k + \cdots + a_2 + a_1 = \abs{G}.
            \end{equation*}
            The constraint \eqref{unity-rep} forces $$1/a_k \le 1/k$$ (or else $1 = k/a_k > k/k = 1$, which is absurd). By Lagrange's theorem, we must also have $a_i | \abs{G}$. I claim these constraints have only finitely many integer solutions for the $a_i$. Since isomorphisms preserve conjugacy classes, we deduce there are only finitely many finite groups (up to isomorphism) with $k$ conjugacy classes.
            
        \item 
          \begin{itemize}
              \item Only the trivial group has exactly $1$ conjugacy class, as the identity is in its own class.
              \item The only integer solution for $a_1$ and $a_2$ given the above constraints is $a_1 = 2$ and $a_2 = 2$. Hence the identity is centralized by exactly $2$ elements, itself and some $g$. But then $g$ is centralized by itself and the identity. So any group with $2$ conjugacy classes is isomorphic to the (unique) group with two elements $\Z/2\Z$.
              \item Suppose $G$ is a group with $3$ conjugacy classes. Suppose $N \in \N$ is a natural number with $3$ positive divisors such that each of these positive divisors is less than $N$ and the sum of these divisors is $N$. Then $N = 6 = 3 + 2 + 1$. Hence $1 = 1/2 + 1/3 + 1/6$ gives the only integer solution for $a_1, a_2$ and $a_3$. Because the conjugation action defines an automorphism from $G$ to $G$, conjugate elements have the same order. Now $S_3$ has $1$ element of order $1$, $2$ elements of order $3$, and $3$ elements of order $2$. Any other finite group with conjugacy classes of size $1, 2$ and $3$ only would necessarily be isomorphic to $S_3$ by order considerations for elements within the conjugacy classes.
          \end{itemize}
    \end{enumerate}
    \end{proof}
\end{enumerate}

\end{document}
