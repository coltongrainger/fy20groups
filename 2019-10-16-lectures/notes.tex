\documentclass{ccg-notes}

\institution{University of Colorado}
\coursenum{MATH 6270}
\coursename{Theory of Groups}
\semester{Fall 2019}
\teacher{Peter Mayr}
\author{Colton Grainger}
\date{\today}
\email{colton.grainger@colorado.edu}
\thanks{Thanks to \texttt{adebray} for the \LaTeX\ template, which I have forked from \url{https://github.com/adebray/latex_style_files}.}

\begin{document}
\frontstuff

\section{2019-08-26}
revision of group actions;
Fundamental Counting Principle for orbit size;
Lagrange's Theorem

\section{2019-08-28}
centralizer;
normalizer;
center of a p-group;
existence of Sylow subgroups

\section{2019-08-30}
Sylow's Theorem;
Cauchy's Theorem;
free groups;
universal mapping property

\section{2019-09-04}
construction and uniqueness of free group;
rank

\begin{prop}[Free groups and their underlying sets.]
    \label{prop:free_forgetful_adjunction}
    Let $F_1$ and $F_2$ be free groups on the sets $X_1$ and $X_2$ respectively. Then $F_1 \cong F_2$ if and only if $\abs{X_1} = \abs{X_2}$.
\end{prop}

\begin{proof}
First we show $\abs{X_1} = \abs{X_2}$ determines $F_1 \cong F_2$. 
So let $\alpha \colon X_1 \to X_2$ be a bijection. 
Then the embedding 
\[X_1 \xrightarrow{\alpha} X_2 \inj F_2 \qq{extends to a unique group homomorphism} \beta_1 \colon F_1 \to F_2\] 
and similarly the embedding
\[X_2 \xrightarrow{\alpha^{-1}} X_1 \inj F_1 \qq{extends to a unique} \beta_2 \colon F_2 \to F_1.\] 
Notice the restriction of the composition $\beta_2 \beta_1$ from its domain $F_1$ to the underlying set $X_1 \subset F_1$ takes each point $x$ in $X_1$ to itself:
\[\beta_2 \beta_1 (x) = \alpha^{-1}\alpha(x) = x.\] 
That is to say, the following diagram commutes. 
\begin{equation*}
\begin{gathered}\xymatrix@=1em{%
    X_1 \ar[dr]_{\id_{X_1}}\ar@{^{(}->}[rr] & & F_1 \ar[dl]^{\beta_2\beta_1}\\
    & F_1 & 
    }
\end{gathered}
\end{equation*}
By the projective property of $F_1$ over $X_1$, the embedding $X_1 \xrightarrow{\id_{X_1}} F_1$ has a unique extension $F_1 \xrightarrow{\cong} F_1$. Since $\beta_2\beta_1$ already extends $X_1 \xrightarrow{\id_{X_1}} F_1$, it must be that  $\beta_2$ is a left inverse to $\beta_1$. 

Similarly, the restriction of $\beta_1\beta_2$ from its domain $F_2$ to the underlying set $X_2 \subset F_2$ is the identity on $X_2$. Hence the following diagram commutes
\begin{equation*}
\begin{gathered}\xymatrix@=1em{%
    X_2 \ar[dr]_{\id_{X_2}}\ar@{^{(}->}[rr] & & F_2 \ar[dl]^{\beta_1\beta_2}\\
    & F_2 & 
    }
\end{gathered}
\end{equation*}
and one may argue that $\beta_1$ is a left inverse to $\beta_2$. We conclude that $\beta_1$ is an isomorphism with inverse $\beta_2$.

Conversely, suppose the free groups over $X_1$ and $X_2$ are isomorphic. One may argue that the free group functor $F \colon \Set \to \Grp$ is a fully faithful functor (\TODO). In this case, the isomorphism $F_1 \xrightarrow{\cong} F_2$ is induced by a unique bijection $X_1 \xrightarrow{\cong}X_2$. 

\end{proof}
\section{2019-09-06}
presentations;
finitely presented groups;
word problem

\section{2019-09-09}
verbal subgroups;
varieties and HSP

\section{2019-09-11}
free groups in varieties;
Burnside problem


\section{2019-09-13}
subnormal series;
simple groups;
extensions;
semidirect product;
solvable groups

\section{2019-09-16}
commutators;
derived series;
central series;
nilpotent groups

\section{2019-09-18}
upper and lower central series;
unitriangular groups

\section{2019-09-20}
characterizations of nilpotent groups;
Fitting subgroup controlling structure of solvable group

\section{2019-09-23}
Frattini subgroup;
Frattini argument;
nilpotence of finite Frattini subgroup

\section{2019-09-25}
Hall pi-subgroups;
crossed homomorphisms

\section{2019-09-27}
Schur-Zassenhaus Theorem;
existence and conjugacy of complements

\section{2019-09-30}
pi-separable groups;
existence of Hall pi-subgroups in solvable groups

\section{2019-10-02}
transfer 

\end{document}
